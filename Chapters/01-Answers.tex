\chapter{Answers}
\label{cp:answers}
\section{Question 1}
\begin{importantbox}
    You should review and understand the concepts of airfoil lift and drag.
\end{importantbox}

When air is flowing over an airfoil, high-pressure and low-pressure regions develop on different surfaces of the airfoil based on the shape of the airfoil. A well-designed airfoil will produce high-pressure regions underneath the airfoil and low-pressure regions on top.

If one were to sum up or integrate the force caused by the pressure around the airfoil, the result would be a single resultant vector. This resultant vector can further be broken down into two components: a vertical and a horizontal component. The vertical component—usually pointing upwards—is denoted \textit{lift}. The horizontal component—usually pointing downstream of the flow—is denoted \textit{drag}.

\section{Question 2}
\begin{importantbox}
    You should review and understand the concepts of forces and moments in static equilibrium.
\end{importantbox}

A force is any action that affects the motion of an object. The forces acting on an aircraft in static equilibrium are \textit{lift}, \textit{weight}, \textit{thrust}, and \textit{drag}. Lift is the aerodynamic force produced by the wings. Weight is the force due to gravity and counteracts the lift force. Thrust is the force produced by the engines or propellers, and drag is the corresponding resisting force.

Moments are forces acting at a distance away from an axis, characterized by their tendency to cause rotational forces. The moments of an aircraft are the rotational forces along the three different axes, which are perpendicular to each other and centered at the aircraft's center of gravity. The \textit{pitching} moment causes the aircraft's nose to rise and dip. The \textit{rolling} moment is the up-and-down motion of the aircraft's wing tips. The \textit{yawing} moment results in the side-to-side motion of the aircraft's nose.

\section{Question 3}
\begin{importantbox}
    You should review and understand the concepts of icing on aerodynamic structures.
\end{importantbox}

The two major types of icing are \textit{rime} and \textit{glaze} icing. Glaze icing is the most dangerous as the larger droplets of super-cooled water create more complicated build-up shapes. Glaze icing occurs in more intense weather conditions, while rime icing occurs with smaller droplets of water from light rain.

All parts that are directly exposed to the airplane's movement through the air are prone to icing accumulation, such as the wings' leading edges, horizontal and vertical stabilizers, engine intakes, propellers, pitot tubes, static ports, and antennas.

There are two main ways to prevent ice build-up on an aircraft. The first method is anti-icing, which is usually applied before the airplane takes off. Its purpose is to prevent ice build-up The second method is active de-icing. Active de-icing includes thermal heating that melts the ice off the wing, chemical de-icing fluids that can be sprayed onto the aircraft during the flight, and mechanical methods like rubber boots that break off layers of ice. 
